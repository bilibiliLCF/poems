\section*{我想成为一名诗人}
\addcontentsline{toc}{section}{我想成为一名诗人}
\begin{center}
\textit{2024年10月20日至11月15日间某日,2025年9月14日改}\hh 
我想成为一名诗人\par
这样,\par
我笔下的泉便是空游无依\footnote{柳宗元《小石潭记》“潭中鱼可百许头,皆若空游无所依”}\par
我笔下的山便是层峦耸翠\footnote{王勃《滕王阁序》“层峦耸翠,上出重霄”}\par
我笔下的峰便是争高直指\footnote{吴均《与朱元思书》“争高直指,千百成峰”}\par
我笔下的瀑便是飞湍瀑流争喧豗\footnote{李白《蜀道难》“飞湍瀑流争喧豗,砯崖转石万壑雷”}\hh
我想成为一名诗人\par
这样,\par
我的失意便是怀才不遇\par
我的得意便是仰天大笑\footnote{李白《南陵别儿童入京》“仰天大笑出门去,我辈岂是蓬蒿人”}\par
我的喜乐伤悲都融为一体\par
马良的笔写不出我诗句的美\par
李白的酒进不入我诗句的意\par
\vspace{2ex}
我不想只栖息大地\footnote{海德格尔:“人生的本质是诗意的,人是诗意地栖息在大地上的”}

我不想只生活在树上\footnote{卡尔维诺《树上的男爵》。实际上,这句诗是直接取材于浙江省2020年高考满分作文《生活在树上》}

给我谢公屐\footnote{李白《梦游天姥吟留别》“脚著谢公屐,身登青云梯”}

给我鲲鹏

我要扶摇直上九万里\footnote{庄子《逍遥游》}
\end{center}